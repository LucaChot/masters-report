\chapter{Introduction (961)}
\label{firstcontentpage} % start page count here

% This is the introduction where you should introduce your work. In
% general the thing to aim for here is to describe a little bit of the
% context for your work -- why did you do it (motivation), what was the
% hoped-for outcome (aims) -- as well as trying to give a brief overview
% of what you actually did.
%
% It's often useful to bring forward some ``highlights'' into this
% chapter (e.g.\ some particularly compelling results, or a particularly
% interesting finding).
%
% It's also traditional to give an outline of the rest of the document,
% although without care this can appear formulaic and tedious. Your
% call.

% Kubernetes is a widely-used open-source container orchestration system that
% automates the deployment, scaling and management of containerized applications.
% It can run production workloads, having becoming the foundation for many cloud
% computing services like Google Cloud and AWS, while also being portable and
% extensible with a rapidly growing ecosystem of support and tools. Efficient
% scheduling is a critical component of Kubernetes, dictating how deployments
% are best allocated across nodes and ensuring no individual node is overloaded.
% Existing Kubernetes schedulers can be classified according to the input data
% they use to inform scheduling decisions: pod-description vs. telemetry. The
% performance of pod-description based schedulers relies on accurate workload
% estimations which is not always available. Meanwhile, telemetry-based schedulers
% use live information to maximise resource utilisation of nodes and prevent them
% from becoming overloaded. These telemetry-based schedulers typically use
% utilisation metrics which can be overly sensitive and generate false
% notifications of full capacity.
%

Kubernetes has firmly established itself as the leading platform for container
orchestration, experiencing widespread adoption across diverse industries and
organizations of all scales~\cite{kubernetes-adoption-statistics}. From
stateless microservices to complex stateful databases, Kubernetes serves as the
infrastructure backbone for cloud computing services like Google
Cloud~\cite{google-gke} and Azure~\cite{azure-aks}. Its versatility, from
supporting machine learning (ML) deployments~\cite{kubernetes-ai}, edge
computing~\cite{cloudraft} end serverless functions~\cite{knative, openwhisk},
solidifies Kubernetes' position as a cornerstone of the cloud-native industry.

A core function of container orchestration is the efficient scheduling of
containers across a cluster. Kubernetes performs online scheduling, a
particularly challenging subset of the NP-hard task-scheduling problem. In
online scheduling, tasks must be allocated without prior knowledge of future
arrivals~\cite{pruhs2004online}. In this dynamic and resource-intensive
landscape, efficient online scheduling is paramount for both performance and
cost-effectiveness. Poor scheduling decisions can lead to significant
infrastructure waste, directly impacting an organization's bottom line.
Misconfiguration in resource requests and limits is a primary cause of
under-utilized resources or idle nodes, leading to higher billing
\cite{cost-strategies,
bin-packing-and-cost-savings-in-kubernetes-clusters-on-aws}.

Kubernetes was built upon the ideas from Borg~\cite{}, a large-scale cluster
management developed and used by Google. Consequently, Kubernetes' default
scheduler~\cite{kube-scheduler} (\texttt{kube-scheduler}) performs bin-packing
based on the resource requests included in Pod descriptions and heuristic-based
capacities calculated from the Nodes' advertised hardware. The performance
achieved by \texttt{kube-scheduler} therefore depends on accurate estimates of
Pod resource usage. While telemetry-based systems have been proposed to improve
scheduling decisions, they often use expensive deep-ML techniques
\cite{bao2019deep, peng2021dl2} or focuses on a per-Node
level~\cite{yang2019design}. These telemetry-based systems typically augment
existing resource requests or re-schedule already running worklods to less
congested Nodes.

\textsc{Pronto} \cite{grammenos_pronto_2021} offers a promising direction for a
purely telemetry-driven approach. This novel federated, asynchronous, and
memory-limited algorithm schedules allows each compute node to work with a
global view of the system by distributing the knowledge federation workload to
reduce the burden on a central scheduler. However, its direct application within
a Kubernetes scheduling
environment faces significant challenges. \textsc{Pronto}'s original design
assumes zero communication latency and yields only a binary "Reject-Job" signal,
which lacks the granularity for sophisticated Node scoring and fails to account
for significant Pod startup latencies observed in real-world Kubernetes clusters
\cite{qadeer_scaling_2022}. Furthermore, empirical investigations into Linux
Pressure Stall Information (PSI) metrics, a potential replacement for the VMware
vSphere CPU-Ready metric, revealed frequent transient spikes attributed to the
container runtime. This would make it difficult to efficiently distinguish
genuine resource contention from noise when collecting a small set of metrics.
These limitations necessitate a more sophisticated signal that is both
\textbf{comparable} for effective Node ranking and \textbf{reservable} to track
the pending impact of in-flight Pods.

\section{Dissertation Aims and Contributions}

This dissertation aims to apply the theory behind \textsc{Pronto} within the
Kubernetes ecosystem by proposing, implementing, and evaluating \textsc{CARICO} (Italian
for "load"), a novel, purely telemetry-driven scheduling algorithm that
maintains the federated, asynchronous, and memory-limited properties of
\textsc{Pronto} explicitly accounting for communication latency.

The key contributions of this project are:
\begin{itemize}
    \item \textbf{Feasibility Analysis of \textsc{Pronto} for Kubernetes:} An
        investigation into the applicability of \textsc{Pronto}'s principles to
        Kubernetes scheduling, highlighting its flawed assumptions and
        presenting empirical evidence for the challenges of using raw telemetry
        metrics.
    \item \textbf{Proposal of \textsc{CARICO}:} A novel federated, asynchronous,
        and memory-limited scoring algorithm that explicitly accounts for
        communication latency.
    \begin{itemize}
        \item Novel application of Federated Singular Value Decomposition (FSVD)
            to build a local model of recent resource usage, adapting the
            standard SVD subspace merge operation for a lack of mean-centering.
        \item Presentation of a novel \textbf{comparable} capacity signal that scores a Node's
            predicted ability to accept new workloads, considering recent
            resource utilisation across the cluster.
        \item Application of simple signal processing techniques to transform
            the generated capacity signal into a \textbf{reservable} metric,
            enabling schedulers to account for communication latencies.
    \end{itemize}
    \item \textbf{Prototype Implementation and Evaluation of \textsc{CARICO}:}
    \begin{itemize}
        \item Extensive investigation into different telemetry metrics and
            signal processing techniques within the setting of a Kubernetes
            cluster to generate an accurate and precise capacity signal.
        \item Analysis of the correctness of the generated signal and
            exploration of approaches for estimating reservation quantities.
        \item Comprehensive evaluation of \textsc{CARICO}'s overall performance
            under various workloads on a real-world Kubernetes cluster,
            comparing its throughput, Quality of Service (QoS), and performance
            isolation against the industry-standard \texttt{kube-scheduler}.
    \end{itemize}
\end{itemize}

\section{Dissertation Overview}
The remainder of this dissertation is structured as follows:
\begin{itemize}
    \item \textbf{Chapter 2 (Background):} This chapter provides foundational
        knowledge, detailing Kubernetes architecture and its scheduling
        mechanism. It introduces \textsc{Pronto}, explaining its core
        mathematics and highlighting its accuracy in predicting performance
        degradation, while also critically examining its weaknesses that
        necessitate a more sophisticated signal. Finally, it reviews existing
        Kubernetes schedulers to contextualize this work within the current
        landscape.
    \item \textbf{Chapter 3 (Design):} This chapter proposes the novel
        \textsc{CARICO} algorithm, providing detailed proofs and reasoning to
        explain the generation and properties of its capacity signal.
    \item \textbf{Chapter 4 (Implementation):} This chapter outlines the
        structure of the \textsc{CARICO} system within Kubernetes, exploring the
        selection of numerous metrics and signal processing techniques employed
        to generate an accurate and precise signal.
    \item \textbf{Chapter 5 (Evaluation):} This chapter presents a comprehensive
        evaluation of \textsc{CARICO}'s performance under different workloads,
        collecting numerous performance metrics and comparing it against the
        standard \texttt{kube-scheduler}.
    \item \textbf{Chapter 6 (Conclusion and Future Work):} This final chapter
        summarizes the key findings of the dissertation and proposes potential
        directions for future research.
\end{itemize}
