\chapter{Introduction}
\label{firstcontentpage} % start page count here

% This is the introduction where you should introduce your work. In
% general the thing to aim for here is to describe a little bit of the
% context for your work -- why did you do it (motivation), what was the
% hoped-for outcome (aims) -- as well as trying to give a brief overview
% of what you actually did.
%
% It's often useful to bring forward some ``highlights'' into this
% chapter (e.g.\ some particularly compelling results, or a particularly
% interesting finding).
%
% It's also traditional to give an outline of the rest of the document,
% although without care this can appear formulaic and tedious. Your
% call.

Task scheduling is the process of organising, prioritising and allocating tasks
or activities to resources. Task-scheduling is classifed as a NP-hard
problem, thus requiring practical solutions to juggle the quality of scheduling
decisions and the computation required to reach those decisions; resources spent
determing allocations could instead be spent performing the tasks. This
balancing act becomes ever more difficult with online schedulers
\cite{pruhs2004online}: schedulers which receive tasks over time, and must
schedule the tasks without any knowledge of the future. Without entirely knowing
of all tasks to come, the scheduler can't guarantee optimal schedules. As a
result, much research has been focused on finding efficient scheduling
algorithms that guarantee solutions as close to optimal as possible.

Task scheduling as a problem spans multiple domains and scale: OS-level CPU and
IO schedulers (e.g. CFQ and FCFS), workflow orchestrators (Airflow), batch
scheduling workloads in datacenters with thousands of machines. This project
aims to tackle container orchestration within Kubernetes
\cite{kube-page}, an extensible open-source platform for managing
containerized workloads and services. Originally created by Google, and thus
inspired by existing in-house systems \cite{Borg}, it became open-source in
2014.

Kubernetes employs \verb|kube-scheduler|, an In-Tree (default) scheduler that
performs bin-packing, balanced allocation, topology-aware placement and
preemption. In addition, \verb|kube-scheduler| implements the \textbf{Scheduling
Framework}, providing extension points on top of which Out-of-Tree (replacement)
schedulers can extend the scheduler behaviour.

Talk about existing Kubernetes extension. Look at quality of Service plugins and
how the current Kubernetes can't support that with bin-packing. Talk about
rather than trying to blindly optimise for throughput, consider QoS. Look into
how to combine in Serverless Compute / Lambda functions.

- Why Pronto? Pronto is a federated, asynchronous, memory-limited algorithm for
online task scheduling across large-scale networks of hundreds of workers
- Each node executes scheduling decisions on whether to accept an incoming job
independently based on the workload seen thus far. In addition, aggregating
local models can contstruct a holistic view of the system

Talk about how there has been few to no signal based kubernetes plugins. Thus,
this project aimed to transport Pronto into the domain of Kubernetes.

[Give an overview of the system structure: Central scheduler, remote scheduler
and aggregate server]

Talk about modifying the mathematics behind Pronto to get it to work in this
domain. Talk about creating the allocation function, the need for a reserve
function.

To sum up, this project makes the following contributions:
\begin{itemize}
\item Implement a Pronto-based scheduler for Kubernetes\\
    \begin{itemize}
    \item Modify the subspace merge operations to take into account a different
    vector space
    \item Introduce signal processing techniques to calculate reserve amount
    \item Implement TCP flow-control inspired technique to push throughput while
    still ensuring QoS
    \end{itemize}
\item Demonstrate that this prototype can achieve competitive throughput with
substantial reduction in per-pod completion time.
\end{itemize}

Potentially discuss the drawbacks.

TODO: Check if there are any existing QoS schedulers that do not use predefined
constraints. What is a use case in which one does not know exactly how much
resources a task will take.

In Chapter 2, we go over existing Kubernetes schedulers and why they do not
achieve true online QoS (no prior knowledge of pod resource usage and
requirement). In Chapter 3, we explain our proposed system: outlining the
mathematics, the system design, allocation algorithm and optimisations. In
Chapter 4, we evaluate the performance of Pronto against the standard
\verb|kube-scheduler|, showing how it greatly reduces individual pod completion
time. Lastly, we conclude our work and propose potential future directions.
