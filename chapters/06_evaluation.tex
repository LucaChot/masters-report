\chapter{Evaluation}
%
% For any practical projects, you should almost certainly have some kind
% of evaluation, and it's often useful to separate this out into its own
% chapter.

\section{Batch workloads}
\begin{tcolorbox}[boxsep=0mm,left=2.5mm,right=2.5mm]
    \textbf{Batch Workloads:} {\em In this section, I will compare Pronto
    against the Kubernetes scheduler when running batch jobs. In this section I
    will explore both CPU, Memory-Intensive and Multi-Resource workloads. I will
    evaluate the throughput, completion time and response time.}
\end{tcolorbox}

\section{Sensitive Workloads}
\begin{tcolorbox}[boxsep=0mm,left=2.5mm,right=2.5mm]
    \textbf{Sensitive Workloads:} {\em In this section I will investigate how
    this scheduler protects sensitive workloads. I will have a server running in
    the background on a node and will schedule a batch of jobs. I will measure
    how a servers response time changes.}
\end{tcolorbox}

\section{Limitation}
\begin{tcolorbox}[boxsep=0mm,left=2.5mm,right=2.5mm]
    \textbf{Limitations:} {\em In this section, I will go over the limitations
    of the system. I will highlight how certain metrics like CPU-Utilisations
    don't give any more information once saturated. I will also have to mention
    how due to the sub-linear pod completion time, the Kubernetes scheduler is
    able to achieve higher job throughput by packing more pods into nodes.
    }
\end{tcolorbox}

\section{Summary}
\begin{tcolorbox}[boxsep=0mm,left=2.5mm,right=2.5mm]
    \textbf{Summary:} {\em In this section, I will summarise the results of my
    evaluation section, highlighting key findings and reasoning.
    }
\end{tcolorbox}

