\chapter{Required Lemmas for Model Interpretation}

\section{Proof: The First Left Singular Vector as a Pseudo Weighted Average}
Let $\mathbf{A}$ be an $m \times n$ matrix with non-negative elements $0 \leq
a_{ij} \leq 1$. Let $u_1$ be the first left singular vector and $\sigma_1$ the
corresponding singular value of $\mathbf{A}$.

\subsection{Fundamental Representation of $u_1$}
The vector $u_1$ is an eigenvector of $\mathbf{AA}^T$ associated with the
largest eigenvalue $\sigma_1^2$: $\mathbf{AA}^T = \sigma_1^2 u_1$. Let $a_j$
denote the $j$-th column of $\mathbf{A}$. Then $\mathbf{AA}^T = \sum_{j=1}^n a_j
a_j^T$. Substituting this into the eigenvalue equation and assuming $\sigma_1
\neq 0$: $\big( \sum_{j = 1}^n a_j a_j^T\big) u_1 = \sigma_1^2 u_1 \implies
\sum_{j=1}^n a_j (a_j^T u_1) = \sigma_1^2 u_1$. Thus $u_1$ can be expressed as a
linear combination of the columns of $\mathbf{A}$:
\[ u_1 = \sum_{j=1}^n \bigg(\frac{a_j^T u_1}{\sigma_1^2}\bigg) a_j =
\sum_{j=1}^n w_j a_j, \text{  where } w_j = \frac{a_j^T u_1}{\sigma_1^2} \]

\subsection{Contribution of Columns to $u_1$}
The magnitude of each weight $w_j$ is given by:
\[ |w_j| = \frac{|a_j^T u_1|}{\sigma_1^2} = \frac{\parallel a_j\parallel
|\cos(\theta_j)|}{\sigma_1^2}\]
where $\theta_j$ is the angle between column $a_j$ and $u_1$. Columns $a_j$ with
larger norms $\parallel a_j \parallel$ or those more parallel to $u_1$ (i.e.
$|\cos(\theta_j)| \approx 1$) will have weights $w_j$ of greater magnitude.
Consequently, these "larger" or more aligned columns contribute more
significantly to the sum defining $u_1$ and thus to its direction

\subsection{Non-Uniqueness of $u_1$ and Relation to Perron-Frobenius Theorem}
The singular vectors in SVD are not unique: if $\mathbf{A} =
\mathbf{U}_1\Sigma\mathbf{V}_1^T = \mathbf{U}_2\Sigma\mathbf{V}_2^T$ then
$\Sigma_1 = \Sigma_2$ but $\mathbf{U}_1 = \mathbf{U}_2\mathbf{B}_a$ and
$\mathbf{V}_1 = \mathbf{V}_2\mathbf{B}_b$ for some block diagonal unitary
matrices $\mathbf{B}_a, \mathbf{B}_b$~\cite{eftekhari2019moses}.

Since $\mathbf{A}$ has non-negative entries ($a_{ij} \geq 0$), $\mathbf{AA}^T$
is a non-negative matrix. By the Perron-Frobenius theorem, there exists an
eigenvector $u_1^+$ corresponding to the eigenvalue $\sigma_1^2$ whose
components $u_{1i}^+$ are all non-negative ($u_{1i}^+ \geq 0$). The first left
singular vector $u_1$ obtained from SVD must then be $u_1 = bu_1^+$ where $b =
\pm 1$

\subsection{Interpretation as a Pseudo Weighted Average}
\begin{itemize}
    \item \textbf{Case 1:} $b = 1 \implies u_1 = u_1^+$.\\
    In this case, $a_j^Tu_1 = a_j^Tu_1^+ = \sum_i a_{ij}u_{1i}^+ \geq 0$ (as
    $a_{ij} \geq 0, u_{1i}^+ \geq 0$).\\
    The weights $w_j = (a_j^Tu_1)/\sigma_1^2$ are therefore non-negative ($w_j
    \geq 0$).\\
    $u_1 = \sum w_ja_j$ becomes a conic combination of the non-negative column
    vectors $a_j$. This $u_1$ (itself non-negative) acts as a pseudo weighted
    average, representing a principal direction within the cone spanned by the
    $a_j$. Columns contributing larger non-negative $w_j$ pull $u_1$ more
    strongly in their direction.
    \item \textbf{Case 2:} $b = -1 \implies u_1 = -u_1^+$.\\
    Here, $a_j^Tu_1 = a_j^T(-u_1^+) =  - a_j^T(u_1^+) \leq 0$.\\
    The weights $w_j = (a_j^Tu_1)/\sigma_1^2$ are non-positive ($w_j \leq 0$).\\
    While $u_1 = \sum w_ja_j$ now involves non-positive weights for non-negative
    vectors $a_j$, the magnitudes $|w_j| = (a_j^Tu_1)/\sigma_1^2$ remain the
    same as in Case 1. Thus, columns $a_j$ that are ``larger" or more aligned
    with $u_1^+$ still contribute with greater magnitude to the sum, determining
    the orientation of the line spanned by u.
\end{itemize}

\subsection{Conclusion}
The vector $u_1$ is a linear combination of the columns of $\mathbf{A}$, $u_1 =
\sum w_j a_j$. The magnitude of the coefficient $w_j$ for each column $a_j$ is
proportional to the projection of $a_j$ onto $u_1$ (scaled by $1/\sigma_1^2$).
This means the columns with larger norms or those more aligned with the
principal direction (the line spanned by $u_1$) contribute more significantly to
defining this direction.

Irrespective of the sign $b$ (determined by the SVD algorithm), the line along
which $u_1$ lies is shaped by this weighted aggregation.

\section{Proof: Monotonicity of the First Singular Value for Non-Negative Matrices}
Let $\mathbf{A}$ and $\mathbf{B}$ be $m \times n$ matrices with real entries
such that $0 \leq a_{ij} \leq b_{ij} \leq 1$ for all $i,j$. We want to show that
$\sigma_1(\mathbf{A}) \leq \sigma_1(\mathbf{B})$, where $\sigma_1(\mathbf{M})$
denotes the first (largest) singular value of a matrix $\mathbf{M}$.

\subsection{Relating Singular Values to $\mathbf{M}^T\mathbf{M}$}
\label{sec:mono-one}
The square of the first singular value, $\sigma_1(\mathbf{M})^2$, is the largest
eigenvalue of the matrix $\mathbf{M}^T\mathbf{M}$. Since
$\mathbf{M}^T\mathbf{M}$ is a symmetric positive semi-definite matrix, its
largest eigenvalue is also its spectral radius, $\rho(\mathbf{M}^T\mathbf{M})$.
Thus, $\rho(\mathbf{A})^2 = \rho(\mathbf{A}^T\mathbf{A})$ and
$\rho(\mathbf{B})^2 = \rho(\mathbf{B}^T\mathbf{B})$.

\subsection{Comparing $\mathbf{A}^T\mathbf{A}$ and $\mathbf{B}^T\mathbf{B}$}
Let $\mathbf{M}_{\mathbf{A}} = \mathbf{A}^T\mathbf{A}$ and
$\mathbf{M}_{\mathbf{B}} = \mathbf{B}^T\mathbf{B}$. The $(j,k)$-th element of
thess matrices are:
\begin{align}
    (\mathbf{M}_{\mathbf{A}})_{jk} &= \sum_{i=1}^m a_{ij}a_{ik} \\
    (\mathbf{M}_{\mathbf{B}})_{jk} &= \sum_{i=1}^m b_{ij}b_{ik}
\end{align}

Since $0 \leq a_{ij} \leq b_{ij}$ for all $i,j$:
\begin{itemize}
    \item All $a_{ij}$ and $b_{ij}$ are non-negative
    \item Therefore, $a_{ij}a_{ik} \leq b_{ij}b_{ik}$ for all $i,j,k$. Summing
        over $i$:
        \[ \sum_{i=1}^m a_{ij}a_{ik} \leq \sum_{i=1}^m b_{ij}b_{ik} \]
        This implies $(\mathbf{M}_{\mathbf{A}})_{jk} \leq
        (\mathbf{M}_{\mathbf{B}})_{jk}$ for all $j,k$. Thus, $0 \leq
        \mathbf{M}_{\mathbf{A}} \leq \mathbf{M}_{\mathbf{B}}$ element-wise. Both
        $\mathbf{M}_{\mathbf{A}}$ and $\mathbf{M}_{\mathbf{B}}$ are matrices
        with non-negative entries.
\end{itemize}

\subsection{Monotonicity of Spectral Redius}
\label{sec:mono-three}
A standard result from Perron-Frobenius theory states that if $\mathbf{X}$ and
$\mathbf{Y}$ are non-negative matrices such that $\mathbf{X} \leq \mathbf{Y}$
element-wise, then their spectral radii satisfy $\rho(\mathbf{X}) \leq
\rho(\mathbf{Y})$. Applying this result to $\mathbf{M}_{\mathbf{A}} =
\mathbf{A}^T\mathbf{A}$ and $\mathbf{M}_{\mathbf{B}} =
\mathbf{B}^T\mathbf{B}$:
\[ \rho(\mathbf{A}^T\mathbf{A}) \leq \rho(\mathbf{B}^T\mathbf{B}) \]

\subsection{Conclusion}
From Section \ref{sec:mono-one} and \ref{sec:mono-three}:
\[ \sigma_1(\mathbf{A})^2 \leq \sigma_1(\mathbf{B})^2 \]
Since the singular values are by definition non-negative ($\sigma_1(\mathbf{M})
\geq 0$):
\[ \sigma_1(\mathbf{A}) \leq \sigma_1(\mathbf{B}) \]
This shows that if the values of a non-negative matrix $\mathbf{A}$ are
increased (while remaining non-negative, e.g., elements between $0$ and $1$),
the first singular value of the resulting matrix $\mathbf{B}$ wil greater than
or equal to that of $\mathbf{A}$.

\section{Proof: Singular Value of Concatenated Matrix}
This section proves the following statement:\\
Given two non-negative matrices $\mathbf{A} \in \mathbb{R}^{m \times n_1}$ and
$\mathbf{B} \in \mathbb{R}^{m \times n_2}$, the first singular value of the
concatenated matrix $\mathbf{C} = [\mathbf{A}, \mathbf{B}]$ is greater than or
equal to the maximum of the first singular values of $\mathbf{A}$ and
$\mathbf{B}$. That is,
\[ \sigma_1([\mathbf{A}, \mathbf{B}]) \geq \max(\sigma_1(\mathbf{A}),
\sigma_1(\mathbf{B})) \]

\subsection{Setup}
Let $\mathbf{A}$ be an $m \times n_1$ matrix and $\mathbf{B}$ be an $m \times
n_2$ matrix. The concatenated matrix $\mathbf{C} = [\mathbf{A}, \mathbf{B}]$ is
an $m \times (n_1 + n_2)$ matrix. The first singular value of any matrix
$\mathbf{M}$, denoted $\sigma_1(\mathbf{M})$, is defined as its spectral norm:
\[ \sigma_1(\mathbf{M}) = \parallel \mathbf{M} \parallel_2 = \max_{\parallel x
\parallel_2 = 1} \parallel \mathbf{M}x \parallel_2 \]
where $x$ is a unit vector.

\subsection{Relating $\sigma_1(\mathbf{C})$ to $\sigma_1(\mathbf{A})$}
\label{sec:bigger-than-A}
Let $v_{\mathbf{A}}^*$ be a unit vector in $\mathbb{R}^{n_1}$ such that
$\sigma_1(\mathbf{A}) = \parallel \mathbf{A}v_{\mathbf{A}}^* \parallel_2$. Such
a vector $v_{\mathbf{A}}^*$ is the right singular vector corresponding to
$\sigma_1(\mathbf{A})$.

Consider a vector $x_0 \in \mathbb{R}^{n_1 + n_2}$ constructs as $x_0 =
\begin{bmatrix} v_{\mathbf{A}}^* \\ 0_{n_2 \times 1} \end{bmatrix}$, where
$0_{n_2 \times 1}$ is the zero vector of dimensions $n_2$.\\
The squared norm of $x_0$ is $\parallel x_0 \parallel_2^2 = \parallel
v_{\mathbf{A}}^*$ \parallel_2^2 + \parallel 0_{n_2 \times 1} \parallel_2^2 = 1^2
+ 0  = 1. Thus, $x_0$ is a unit vector.

By definitions of $\sigma_1(\mathbf{C})$:
\[ \sigma_1(\mathbf{C}) = \max_{\parallel x \parallel_2 = 1} \parallel
\mathbf{C}x \parallel_2 \]
Since $x_0$ is a specific unit vector, we have:
\[ \sigma_1(\mathbf{C}) \geq \parallel \mathbf{C}x_0 \parallel_2 \]
Computing $\mathbf{C}x_0$ gives:
\[ \mathbf{C}x_0 = [\mathbf{A}, \mathbf{B}] \begin{bmatrix} v_{\mathbf{A}}^* \\
0_{n_2 \times 1} \end{bmatrix} = \mathbf{A}v_{\mathbf{A}}^* + \mathbf{B} \dot
0_{n_2 \times 1} = \mathbf{A}v_{\mathbf{A}}^* \]
Therefore,
\[ \sigma_1(\mathbf{C}) \geq \parallel \mathbf{A}v_{\mathbf{A}}^* \parallel_2 =
\sigma_1(\mathbf{A})\]

\subsection{Relating $\sigma_1(\mathbf{C})$ to $\sigma_1(\mathbf{B})$}
\label{sec:bigger-than-B}
Similarly, let $v_{\mathbf{B}}^*$ be a unit vector in $\mathbb{R}^{n_2}$ such that
$\sigma_1(\mathbf{B}) = \parallel \mathbf{B}v_{\mathbf{B}}^* \parallel_2$. Such

Consider a vector $x_1 \in \mathbb{R}^{n_1 + n_2}$ constructs as $x_1 =
\begin{bmatrix} 0_{n_1 \times 1} \\ v_{\mathbf{B}}^* \end{bmatrix}$.\\
The squared norm of $x_1$ is $\parallel x_1 \parallel_2^2 = \parallel 0_{n_1
\times 1} \parallel_2^2 + \parallel v_{\mathbf{B}}^*$ \parallel_2^2 = 0 + 1^2 =
1. Thus, $x_1$ is a unit vector.

Again, by definitions of $\sigma_1(\mathbf{C})$:
\[ \sigma_1(\mathbf{C}) \geq \parallel \mathbf{C}x_1 \parallel_2 \]
Computing $\mathbf{C}x_1$ gives:
\[ \mathbf{C}x_1 = [\mathbf{A}, \mathbf{B}] \begin{bmatrix} 0_{n_1 \times 1} \\
v_{\mathbf{B}}^* \end{bmatrix} = \mathbf{A} \dot 0_{n_1 \times 1} + \mathbf{B}
v_{\mathbf{B}}^* = \mathbf{B}v_{\mathbf{B}}^* \]
Therefore,
\[ \sigma_1(\mathbf{C}) \geq \parallel \mathbf{B}v_{\mathbf{B}}^* \parallel_2 =
\sigma_1(\mathbf{B})\]

\subsection{Conclusion}
From Sections \ref{sec:bigger-than-A} and \ref{sec:bigger-than-B}, we have shown
that $\sigma_1(\mathbf{C}) \geq \sigma_1(\mathbf{A})$ and $\sigma_1(\mathbf{C})
\geq \sigma_1(\mathbf{B})$. This implies that $\sigma_1(\mathbf{C})$ must be
greater than or equal to the maximum of these two values:
\[ \sigma_1([\mathbf{A}, \mathbf{B}]) \geq \max(\sigma_1(\mathbf{A}),
\sigma_1(\mathbf{B})) \]

\section{$\sigma_1$ of Scaled Concatenation}
This section proves the following statement:
Given two non-negative matrices $\mathbf{A} \in \mathbb{R}^{m \times n_1}$ and
$\mathbf{B} \in \mathbb{R}^{m \times n_2}$ and non-negative scalar weights
$\gamma_{\mathbf{A}} = \sqrt{w_{\mathbf{A}}}$ and $\gamma_{\mathbf{B}} =
\sqrt{w_{\mathbf{B}}}$ such that $w_{\mathbf{A}} + w_{\mathbf{B}} = 1$, the
resulting first singular value and the first left
singular vector of the concatenated matrix $\mathbf{C} = [\gamma_{\mathbf{A}}\mathbf{A},
\gamma_{\mathbf{B}}\mathbf{B}]$ has the following property:
\[ \min(P_{\mathbf{A}}(u_{\mathbf{C}}),P_{\mathbf{B}}(u_{\mathbf{C}})) \leq \sigma_{\mathbf{C}}^2 \leq
\max(P_{\mathbf{A}}(u_{\mathbf{C}}),P_{\mathbf{B}}(u_{\mathbf{C}})) \]
where $\sigma_{\mathbf{C}}$ and $u_{\mathbf{C}}$ correspond to the first singular value and first
left singular vector of $\mathbf{C}$, $P_{\mathbf{M}}(u)$ is the sum of squared scalar porjections
of the columns in $\mathbf{M}$ onto $u$ (let $m_j$ be the $j$-th column of $\mathbf{M}$ $P_{\mathbf{M}}(u)
= \sum_{j=1}^n (u^T m_j)^2$).

\subsection{Expanding $\sigma_1(C)$}
By definition, $S_{C} = \sigma_1(C)$ and $u_{C}$ is the corresponding first
singular left vector. Thus, they satisfy the eigenvalue equation for $CC^T$:
\[ CC^Tu_{C} = S_{C}^2 u_{C} \]
Pre-multiplying by $u_{C}^T$:
\begin{align}
    u_{C}^T(CC^T u_{C}) &= u_{C}^T(S_{C}^2 u_{C}) \\
    u_{C}^T CC^T u_{C} &= S_{C}^2 (u_{C}^T u_{C})
\end{align}
Since $u_{C}$ is a unit vector, $u_{C}^T u_{C} = \parallel u_{C} \parallel_2^2 =
1$. So,
\[ S_{C}^2 = u_{C}^T CC^T u_{C} \]
Now, lets express $CC^T$ in terms of $A$ and $B$:\\
The matrix $C = [\gamma_{A}A,\gamma_{B}B]$.\\
Then $C^T = \begin{bmatrix} \gamma_{A}A^T \\ \gamma_{B}B^T \end{bmatrix}$.\\
So, $CC^T = [\gamma_{A}A,\gamma_{B}B]\begin{bmatrix} \gamma_{A}A^T \\
\gamma_{B}B^T \end{bmatrix} = (\gamma_{A}A)(\gamma_{A}A^T) +
(\gamma_{B}B)(\gamma_{B}B^T)$
\[ CC^T = \gamma_{A}^2AA^T + \gamma_{B}^2BB^T \]




 is greater than or equal to the maximum of the first singular
values of $\mathbf{A}$ and $\mathbf{B}$. That is,
\[ \sigma_1([\mathbf{A},
\mathbf{B}]) \geq \max(\sigma_1(\mathbf{A}), \sigma_1(\mathbf{B})) \]
$\sigma_1(\mathbf{C})$




% Some traditional fake Latin prose for calibrating words/page
Lorem ipsum dolor sit amet, consetetur sadipscing elitr, sed diam
nonumy eirmod tempor invidunt ut labore et dolore magna aliquyam erat,
sed diam voluptua. At vero eos et accusam et justo duo dolores et ea
rebum. Stet clita kasd gubergren, no sea takimata sanctus est Lorem
ipsum dolor sit amet. Lorem ipsum dolor sit amet, consetetur
sadipscing elitr, sed diam nonumy eirmod tempor invidunt ut labore et
dolore magna aliquyam erat, sed diam voluptua. At vero eos et accusam
et justo duo dolores et ea rebum. Stet clita kasd gubergren, no sea
takimata sanctus est Lorem ipsum dolor sit amet. Lorem ipsum dolor sit
amet, consetetur sadipscing elitr, sed diam nonumy eirmod tempor
invidunt ut labore et dolore magna aliquyam erat, sed diam voluptua.
At vero eos et accusam et justo duo dolores et ea rebum. Stet clita
kasd gubergren, no sea takimata sanctus est Lorem ipsum dolor sit
amet.

