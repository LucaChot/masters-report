\chapter*{Abstract}

Pronto is a federated asynchronous, memory-limited algorithm proposed for online
task-scheduling across large-scale networks of hundreds of workers. Each
individual node to update their own local model basd on the workload seen so far
and generate a rejection signal which reflects the overall node responsiveness
and whether it can accept an incoming task. In addition, aggregating the local
models builds a global view of the system. Kubernetes is an existing open-source
container orchestration system that automates the deployment, scaling and
management of containerized applications. It can run production workloads,
having becoming the foundation for many cloud computing services like Google
Cloud and AWS, while also being portable and extensible with a rapidly growing
ecosystem of support and tools.

The standard Kubernetes scheduler uses fixed node and pod information to make
scheduling decisions. I propose to apply Pronto to the Kubernetes ecosystem. The
aim is to implement an online scheduler that uses online methods to both
maximies resource utilisation and minimise per-pod latency while still being
competitive with the industry-standard Kubernetes scheduler. During this
project I had to modify the mathematics behind Pronto, proposing novel sub-space
merging techniques, as well as a new signal function. In addition, I implemented
a prototype of the Pronto-based Kubernetes scheduler, comparing it against the
standard Kubernetes scheduler. These experiments demonstrate that Pronto is
able to achieve a significantly lower pod completion time distribution while
remaining competitive (Include values). Thus, Pronto may exceed in situations
were pod completion is weighted more than throughput of pods, such as serverless
compute.
