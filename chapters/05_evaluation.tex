\chapter{Evaluation}
%
% For any practical projects, you should almost certainly have some kind
% of evaluation, and it's often useful to separate this out into its own
% chapter.

\section{CPU-centric workloads}
\begin{tcolorbox}[boxsep=0mm,left=2.5mm,right=2.5mm]
    \textbf{CPU-Centric Metric:} {\em In this section, I will compare Pronto
    against the Kubernetes scheduler when running simple CPU-based pods. I
    will measure job completion time, resource utilisation and individual pod
    completion time distribution
    }
\end{tcolorbox}


\section{Short and Long Workloads}
\begin{tcolorbox}[boxsep=0mm,left=2.5mm,right=2.5mm]
    \textbf{Short and Long Workloads:} {\em In this section, I will compare
    Pronto against the kubernetes scheduler when running pods of variable
    run-times.
    }
\end{tcolorbox}

\section{Multi-Resource Workloads}
\begin{tcolorbox}[boxsep=0mm,left=2.5mm,right=2.5mm]
    \textbf{Multi-Resource Workloads:} {\em Finally, I will compare Pronto
    against the kubernetes scheduelr when running pods .
    }
\end{tcolorbox}

\section{Limitation}
\begin{tcolorbox}[boxsep=0mm,left=2.5mm,right=2.5mm]
    \textbf{Limitations:} {\em In this section, I will go over the limitations
    of the system. I will highlight how certain metrics like CPU-Utilisations
    don't give any more information once saturated. I will also have to mention
    how due to the sub-linear pod completion time, the kubernetes scheduler is
    able to achieve higher job throughput by packing more pods into nodes.
    }
\end{tcolorbox}

\section{Summary}
\begin{tcolorbox}[boxsep=0mm,left=2.5mm,right=2.5mm]
    \textbf{Summary:} {\em In this section, I will summarise the results of my
    evaluation section, highlighting key findings and reasoning.
    }
\end{tcolorbox}
