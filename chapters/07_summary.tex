\chapter{Conclusion}

% As you might imagine: summarizes the dissertation, and draws any
% conclusions. Depending on the length of your work, and how well you
% write, you may not need a summary here.
%
% You will generally want to draw some conclusions, and point to
% potential future work.
%

\section{Summary}
This dissertation set out to explore the application of \textsc{Pronto}'s
underlying theory to a Kubernetes scheduler. \textsc{Pronto}'s ability to
efficiently aggregate and process telemetry produced across a datacenter
addressed the limitations of telemtry-focused schedulers which aimed to reduce
the reliance on user-provided resource requests. However, the dynamic nature of
Kubernetes necessitated a novel scorable and reservable signal.

The project successfully achieved its objectives by proposing \textsc{Carico}, a
federated, asynchronous, and memory-limited scoring algorithm that explicitly
accounts for communication latency, a key challenge in real-world Kubernetes
environments. A significant contribution was the novel application of Federated
Singular Value Decomposition (FSVD) on non-mean-centered data to build a local
model of recent resource usage, providing a unique interpretation of resource
utilisation direction and magnitude. This led to the development of a
continuous, comparable, and reservable capacity signal, expressed in terms of
Pod units, which allows for robust reservation mechanisms and dynamic adaptation
to changing workloads.

The comprehensive evaluation of the \textsc{Carico} prototype within a Kubernetes cluster
yielded several key findings. \textsc{Carico} demonstrated comparable Job
Completion times to the default kube-scheduler, crucially without requiring
explicit Pod resource requests. More significantly, it consistently achieved
substantially lower Pod Completion times and improved workload isolation,
highlighting its effectiveness as a Quality of Service (QoS) scheduler. The
evaluation also confirmed CARICO's low overhead, estimated at approximately 2\%,
making it a practical solution for real-world deployments. These results,
combined with \textsc{Carico} portable nature, highlight its potential to
significantly enhance scheduling efficiency and resource utilisation in
Kubernetes and other orchestration systems by providing a more adaptive and
performance-aware approach.

% What is the impact of this paper

\section{Future Work}
The insights gained from \textsc{Carico}'s development and evaluation open several
promising avenues for future research. One significant direction involves
exploring \textsc{Carico}'s applicability in other scheduling environments where explicit
resource requests are either impractical or unavailable. The core design
principles of \textsc{Carico}, particularly its federated, asynchronous, and
memory-limited nature, coupled with its ability to derive meaningful capacity
signals from telemetry, are not intrinsically tied to Kubernetes. This inherent
simplicity suggests that \textsc{Carico} could be adapted for diverse scheduling
challenges beyond container orchestration.

Further research could also focus on expanding the range of metrics \textsc{Carico}
utilizes for its capacity signal generation. Currently, \textsc{Carico} primarily uses
CPU and memory metrics. Incorporating additional telemetry data, such as network
I/O, disk I/O, or GPU utilization, could potentially provide a more holistic and
accurate representation of a Node's workload capacity, leading to even more
optimized scheduling decisions.

Finally, an important area of exploration is to evaluate \textsc{Carico}'s performance on
physical machines rather than virtualized environments. As noted in the
evaluation, hypervisors can mask certain performance degradation effects, making
CPU utilization alone an unreliable indicator of true contention. Testing \textsc{Carico}
on bare-metal servers could reveal how its telemetric-only approach performs
when telemetry data is more directly representative of underlying hardware
contention and actual performance degradation. This would provide valuable
insights into \textsc{Carico}'s robustness and effectiveness in diverse infrastructure
settings.

